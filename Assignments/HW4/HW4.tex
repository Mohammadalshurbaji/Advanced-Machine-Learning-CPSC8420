\documentclass[11pt]{article}
\usepackage{amsmath,amsbsy,amssymb,verbatim,fullpage,ifthen,graphicx,bm,amsfonts,amsthm,url}
\usepackage{graphicx}
\usepackage{xcolor}
\newcommand{\mfile}[1]  {{\small \verbatiminput{./#1}}} % Jeff Fessler, input matlab file
\newcommand{\tmop}[1]{\ensuremath{\operatorname{#1}}}
%\newcommand*{\qed}{\hfill\ensuremath{\blacksquare}}%
\newcommand{\R}{\mathbb{R}}
\newcommand{\C}{\mathbb{C}}
\newcommand{\Z}{\mathbb{Z}}
\newcommand{\A}{\mathcal{A}}
\newcommand{\minimize}{\operatorname*{minimize\ }}
\newcommand{\maximize}{\operatorname*{maximize}}
\newcommand{\opdet}[1]{\operatorname{\textbf{det}}\left(#1\right)}
\newcommand{\optr}[1]{\operatorname{\textbf{tr}}\left(#1\right)}
\newcommand{\AnswerDefine}{}
\newcommand{\answer}[2][blue]{\ifdefined\AnswerDefine{\color{#1}\it#2}\fi}
\newcommand{\mtx}[1]{\mathbf{#1}}
\newcommand{\vct}[1]{\mathbf{#1}}
\def \lg       {\langle}
\def \rg       {\rangle}
\def \mA {\mtx{A}}
\def \mF {\mtx{F}}
\def \mG {\mtx{G}}
\def \mI {\mtx{I}}
\def \mJ {\mtx{J}}
\def \mU {\mtx{U}}
\def \mS {\mtx{S}}
\def \mV {\mtx{V}}
\def \mW {\mtx{W}}
\def \mLambda {\mtx{\Lambda}}
\def \mSigma {\mtx{\Sigma}}
\def \mX {\mtx{X}}
\def \mY {\mtx{Y}}
\def \mZ {\mtx{Z}}
\def \zero     {\mathbf{0}}
\def \vzero    {\vct{0}}
\def \vone    {\vct{1}}
\def \va {\vct{a}}
\def \vg {\vct{g}}
\def \vu {\vct{u}}
\def \vv {\vct{v}}
\def \vx {\vct{x}}
\def \vy {\vct{y}}
\def \vz {\vct{z}}
\def \vphi {\vct{\phi}}
\def \vmu {\vct{\mu}}
\def \R {\mathbb{R}}

%\newcommand{\st}{\operatorname*{\ subject\ to\ }}
\usepackage{algorithm,algpseudocode}
\usepackage{xspace}
% Add a period to the end of an abbreviation unless there's one
% already, then \xspace.
\makeatletter
\DeclareRobustCommand\onedot{\futurelet\@let@token\@onedot}
\def\@onedot{\ifx\@let@token.\else.\null\fi\xspace}

\def\eg{\emph{e.g}\onedot} \def\Eg{\emph{E.g}\onedot}
\def\ie{\emph{i.e}\onedot} \def\Ie{\emph{I.e}\onedot}
\def\cf{\emph{c.f}\onedot} \def\Cf{\emph{C.f}\onedot}
\def\etc{\emph{etc}\onedot} \def\vs{\emph{vs}\onedot}
\def\wrt{w.r.t\onedot} \def\dof{d.o.f\onedot}
\def\etal{\emph{et al}\onedot} \def\st{\emph{s.t}\onedot}
\pagestyle{plain}

\title{{\bf Homework Set 4, CPSC 8420, Fall 2024}} % Change to the appropriate homework number
\author{\Large\underline{Last Name, First Name}}
\date{\textbf{\Large\textcolor{red}{Due 11/19/2024, 11:59PM EST}}} % put your name in the LastName, FirstName format
%\date{\today}

\begin{document}
\maketitle

\section*{Problem 1}
Considering soft margin SVM, where we have the objective and constraints as follows:
\begin{equation}\label{eq:1}
	\begin{aligned}
		min\;\; &\frac{1}{2}||w||_2^2 +C\sum\limits_{i=1}^{m}\xi_i\\s.t.  \;\; y_i(w^Tx_i + &b)  \geq 1 - \xi_i \;\;(i =1,2,...m)\\\xi_i \geq &0 \;\;(i =1,2,...m)
	\end{aligned}
\end{equation}
Now we formulate another formulation as:
\begin{equation}
	\begin{aligned}
		min\;\; &\frac{1}{2}||w||_2^2 +\frac{C}{2}\sum\limits_{i=1}^{m}\xi_i^2\\s.t.  \;\; y_i(w^Tx_i + &b)  \geq 1 - \xi_i \;\;(i =1,2,...m)
	\end{aligned}
\end{equation}
\begin{enumerate}
	\item Different from Eq. (\ref{eq:1}), we now drop the non-negative constraint for $\xi_i$, please show that optimal value of the objective will be the same when $\xi_i$ constraint is removed.
	\item What's the generalized Lagrangian of the new soft margin SVM optimization problem?
	\item Now please minimize the Lagrangian with respect to $w, b$, and $\xi$.
	\item What is the dual of this version soft margin SVM optimization problem? (should be similar to Eq. (10) in the slides)
\end{enumerate}


\end{document}
